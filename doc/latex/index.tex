Multi-\/layer hardware abstraction library for Daisy Product family

On S\+T\+M32\+H7 M\+C\+Us

Lower-\/levels use S\+T\+M32 H\+AL (local copy w/ modifications in Drivers/)

Prefixes and their meanings\+:


\begin{DoxyItemize}
\item sys -\/ System level configuration (clocks, dma, etc.)
\item per -\/ Peripheral level, internal to M\+CU (i2c, spi, etc.)
\item dev -\/ External device support (external flash chips, D\+A\+Cs, codecs, etc.)
\item hid -\/ User level interface elements (encoders, switches, audio, etc.)
\item util -\/ library level elements used within the library (not included via \hyperlink{daisy_8h_source}{daisy.\+h})
\item daisy -\/ core A\+PI files (specific boards, platforms have extended user A\+P\+Is that configure libdaisy more below).
\end{DoxyItemize}

Also included is a core/ folder containing\+:


\begin{DoxyItemize}
\item a generic Makefile that can be included in a project Makefile to simplify getting started
\item a linker script for defining the sections of memory used by the firmware
\item core files for starting the hardware (\hyperlink{system__stm32h7xx_8c}{system\+\_\+stm32h7xx.\+c}, startup\+\_\+stm32h750xx.\+s, etc.) 


\end{DoxyItemize}\hypertarget{index_autotoc_md3}{}\section{Using libdaisy}\label{index_autotoc_md3}
Due to the amount of hardware configuration and flexibility of the daisy platform, (in the present, and the future), a user can use libdaisy to define their own custom hardware, or include one of our supported board files to jumpstart the creativity, and hack on an existing piece of hardware.

If you are getting started, and have one of the Daisy Family Products, you can skip ahead to that section below.\hypertarget{index_autotoc_md4}{}\subsection{daisy.\+h}\label{index_autotoc_md4}
The base-\/level include file. This is all you need to include to create your own custom hardware that uses libdaisy.

{\ttfamily \hyperlink{daisy__seed_8h_source}{daisy\+\_\+seed.\+h}} is an example of a board level file that utilizes libdaisy to define some hardware, and provide flexible access.\hypertarget{index_autotoc_md5}{}\subsection{daisy\+\_\+seed.\+h}\label{index_autotoc_md5}
The S\+O\+M-\/level include file. This can be used with any boards that use the Daisy Seed hardware.

Additional configuration files, with more specific hardware access are provided below for our supported hardware platforms.\hypertarget{index_autotoc_md6}{}\subsection{daisy\+\_\+platform.\+h}\label{index_autotoc_md6}
Several other pairs of files exist in the repo for each of the supported hardware platforms that work with Daisy Seed.

These are\+:
\begin{DoxyItemize}
\item daisy\+\_\+field
\item daisy\+\_\+patch
\item daisy\+\_\+petal
\item daisy\+\_\+pod
\end{DoxyItemize}

With these files a number of additional initialization, and configuration is done by the library.

This allows a user to jump right into their new product with a simple api to do things without having a full understanding of what\textquotesingle{}s going on under the hood. 



With this flexible approach to the hardware configuration, we hope to promote a lot of fantastic hardware along with code to go with it. 